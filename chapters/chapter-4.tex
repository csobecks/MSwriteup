\chapter{Conclusions} \label{ch:conclusion}
In summary, we have presented a stable recursive method for calculating the canonical partition function for a low temperature non-interacting Fermi gas. This method utilizes the Poisson binomial recursion relation (Eq.\@ (\ref{pbrr})) to take the occupation probabilities in the grand canonical ensemble and map them to a canonical ensemble with $N$ particles (Eq.\@(\ref{ZCasZGC})). This method fixes the numerical instabilities that are present in previous methods of calculating the canonical partition function (Eq.\@ (\ref{Borrmann})) and agrees very well with exact solutions (Eq.\@ (\ref{schoneqn})). The agreement with Sch\"onhammer's exact result is shown in Fig.\@ (\ref{fig: schon solution}) with a controllable accuracy shown in Fig.\@ (\ref{fig: schon diff}). With this agreement, we have a method that can take the occupation probabilities measured via TOF in cold atom experiments and extract the temperature in the canonical ensemble. Further, we have developed the theory to describe the difference in temperature measurements between the canonical ensemble and grand canonical ensemble. From here, we have identified regimes where quantum mechanics and statistic ensembles interact to yield potentially large errors when extracting the temperature. Both nondegenerate and degenerate energy spectra were considered for these regimes resulting in theoretical descriptions of these potential errors. These descriptions were shown to agree with the Poisson binomial recursive solutions as shown in Figs. (\ref{fig:linnondeg}-\ref{fig:quadE_almostdegen}).

An indirect future application is to quantum Monte Carlo (QMC). A sign problem exists in (QMC) for fermions similar to the way that a sign problem showed up in Borrmann's equation (\ref{Borrmann}). There may be a similarity between the sources of these sign problems that would allow for a similar method to be developed for QMC in the canonical ensemble. Along with this, the method of getting the canonical ensemble from the grand canonical ensemble may be applied to QMC calculations that are performed in the grand canonical ensemble, but apply to the canonical ensemble. 

Future work with this project would be to collaborate with experimentalists and test the method on a variety of experimental data. Through collaboration, improvements to the speed and numerical stability to the method could be found. It would provide experimentalists accurate results for their data and allow for correction to previously published data sets. Through this collaboration, we hope to make a contribution to the field of statistical mechanics.