\chapter{Conclusions} \label{ch:conclusion}
In summary, I have presented a stable recursive method for calculating the canonical partition function for a low temperature non-interacting Fermi gas. This method utilizes the Poisson binomial recursion relation (Eq.\@ (\ref{pbrr})) to take the occupation probabilies in the grand canonical ensemble and map them to a canonical ensemble with $N$ particles (Eq.\@(\ref{Ztildetopb}). This method fixes the numerical instabilities that are present in previous methods of calculating the canonical partition function (Eq.\@ (\ref{Borrmann})) and agrees very well with exact solutions (Eq.\@ (\ref{schoneqn})). The agreement with Sch\"onhammer's exact result is shown in Fig.\@ (\ref{fig: schon solution}) with a controllable accuracy shown in Fig.\@ (\ref{fig: schon diff}). With a proper benchmark, this method is used to extract the error in temperature which is shown in Figs. (\ref{fig:linnondeg}-\ref{fig:quadE_almostdegen}). For the nondegenerate and degenerate cases, the theory and program agree very well proving the accuracy of each. 

Future work with this project would be to collaborate with experimentalists test the program on a variety of experimental data. This would be a great testing ground to see the full capability of the program. Through collaboration, improvements to the speed and numerical stability to the program could be found. It would give the experimentalists accurate results for their data and allow for correction to previously published data sets.

An indirect future application is to quantum Monte Carlo (QMC). A sign problem exists in (QMC) for fermions similar to the way that a sign problem showed up in Borrmann's equation (\ref{Borrmann}). There may be a similarity between the sources of these sign problems that would allow for a similar method to be developed for QMC. Along with this, the method of getting the canonical ensemble from the grand canonical ensemble may be applied to QMC calculations that are performed in the grand canonical ensemble, but apply to the canonical ensemble. 