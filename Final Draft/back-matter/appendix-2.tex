\chapter{$\chi^2$ derivatives}
The function, $\chi^2$, is defined as follows.
\begin{equation}
\chi^2=\sum_i(p_i-\avg{n_c}_i)^2
\end{equation}
The first derivative can be set to zero to find the minima and the second derivative will be used to determine if the zero values are at the the true minimum or at the tail end of this function. Solving the first derivative is done as follows.
\begin{gather}
    d\chi^2=\eval{\frac{\partial \chi^2}{\partial \beta^*}}_y d \beta^* 
    +\eval{\frac{\partial \chi^2}{\partial y}}_{\beta^*} d y\\    
    \frac{d\chi^2}{d\beta^*}=\eval{\frac{\partial \chi^2}{\partial 
    \beta^*}}_y +\frac{\partial \chi^2}{\partial y}\Biggr|_{\beta^*} 
    \frac{d y}{d \beta^*}
\end{gather}
The condition that the sum of $p_i$ must equal the total number of particles can be utilized to find the last term in the equation above. 
\begin{gather}
    dN=0=\eval{\frac{\partial N}{\partial \beta^*}}_y d \beta^* +\eval{\frac{\partial N}{\partial y}}_{\beta^*} d y\\
    \eval{\frac{\partial N}{\partial \beta^*}}_y d \beta^*=-\eval{\frac{\partial N}{\partial y}}_{\beta^*} d y\nonumber\\
    \frac{d y}{d \beta^*}=-\frac{\eval{\frac{\partial N}{\partial \beta^*}}_y}{\eval{\frac{\partial N}{\partial y}}_{\beta^*}}\\
    \eval{\frac{\partial N}{\partial \beta^*}}_y= \sum_i \frac{\partial p_i}{\partial \beta^*}=\sum_i \frac{-e^{\beta^* \epsilon_i-y}}{(1+e^{\beta^* \epsilon_i-y})^2}\epsilon_i=-\sum_i \epsilon_i p_i q_i\\
    \eval{\frac{\partial N}{\partial y}}_{\beta^*}=\frac{\partial p_i}{\partial y}=\sum_i -\frac{-e^{\beta^* \epsilon_i-y}}{(1+e^{\beta^* \epsilon_i-y})^2}=\sum_i p_i q_i\\
    \frac{d y}{d \beta^*}=-\frac{-\sum_i \epsilon_i p_i q_i}{\sum_i p_i q_i}=\frac{\sum_i \epsilon_i p_i q_i}{\sum_i p_i q_i}=C_1
\end{gather}
Turning back to the original equation, 
\begin{align}
    \eval{\frac{\partial \chi^2}{\partial \beta^*}}_y&=\sum_i 2(p_i-\avg{n_c}_i) \frac{-\epsilon_i e^{\beta^* \epsilon_i-y}}{1+e^{\beta^* \epsilon_i-y}}=\sum_i -2(p_i-\avg{n_c}_i)p_i q_i \epsilon_i\\
    \eval{\frac{\partial \chi^2}{\partial y}}_{\beta^*}&=\sum_i 2(p_i-\avg{n_c}_i) \frac{e^{\beta^* \epsilon_i-y}}{1+e^{\beta^* \epsilon_i-y}}=\sum_i 2(p_i-\avg{n_c}_i)p_i q_i\\
    0&=\frac{d\chi^2}{d\beta^*}=\sum_i -2(p_i-\avg{n_c}_i)p_i q_i \epsilon_i+\sum_i 2(p_i-\avg{n_c}_i)p_i q_i C_1\nonumber\\
    0&=\sum_i 2(p_i-\avg{n_c}_i)p_i q_i (\epsilon_i-C_1)\nonumber\\
    0&=\sum_i 2(p_i-\avg{n_c}_i)p_i q_i (\epsilon_i-\frac{\sum_k \epsilon_k p_k q_k}{\sum_k p_k q_k} )
\end{align}
The equation for the first derivative can be set equal to zero in order to find the minimum.

Before looking at the second derivative, it is beneficial to analyze the derivatives of the functions $p_i$ and $q_i$.
\begin{gather}
    \eval{\frac{\partial p_i}{\partial \beta^*}}_y=-\epsilon_i p_i q_i\\
    \eval{\frac{\partial p_i}{\partial y}}_{\beta^*}=p_i q_i\\
    \eval{\frac{\partial q_i}{\partial \beta^*}}_y=\frac{\epsilon_i e^{y- 
    \beta^* \epsilon_i}}{(e^{y-\beta^* \epsilon_i}+1)^2}=q_i p_i 
    \epsilon_i\\
    \eval{\frac{\partial q_i}{\partial y}}_{\beta^*}=\frac{-e^{y-\beta^* 
    \epsilon_i}}{(e^{y-\beta^* \epsilon_i}+1)^2}=-q_i p_i
\end{gather}
These will be useful for the calculations of the second derivative which is done as follows.  
\begin{gather}
    \frac{d}{d\beta^*} \qty{\frac{d \chi^2}{d\beta^*}} = \frac{\partial } 
    {\partial \beta^*} \eval{\qty{\frac{d \chi^2}{d\beta^*}}}_y+ 
    \frac{\partial }{\partial y} \eval{\qty{\frac{d \chi^2} 
    {d\beta^*}}}_{\beta^*} \frac{dy}{d\beta^*}\nonumber\\
    \frac{d}{d\beta^*} \qty{\frac{d \chi^2}{d\beta^*}} = \frac{\partial } 
    {\partial \beta^*} \eval{\qty{\frac{d \chi^2}{d\beta^*}}}_y+ 
    \frac{\partial }{\partial y} \eval{\qty{\frac{d \chi^2} 
    {d\beta^*}}}_{\beta^*} C_1
\end{gather}
Once again, each component needs to be calculated. The term $C_1$ is the same so the other two terms are the only ones to be calculated. Starting on the left equation,
\begin{align}
&\frac{\partial}{\partial \beta^*} \qty{\frac{d \chi^2}{d \beta^*}} =\frac{\partial}{\partial \beta^*} [-2\sum_i (p_i-\avg{n_c}_i)p_i q_i \epsilon_i+2\sum_i (p_i-\avg{n_c}_i)p_i q_i C_1]\\
    &\quad=-2\sum_i[p_i q_i \epsilon_i (-\epsilon_i p_i q_i) +(p_i-\avg{n_c}_i) q_i \epsilon_i(-\epsilon_i p_i q_i)+ (p_i-\avg{n_c}_i)p_i \epsilon_i(q_i p_i \epsilon_i)\nonumber\\
    &\quad+(-\epsilon_i p_i q_i) p_i q_i C_1+ (p_i-\avg{n_c}_i)(-\epsilon_i p_i q_i) q_i C_1+(p_i-\avg{n_c}_i)p_i(q_i p_i \epsilon_i) C_1\nonumber\\
    &\quad+ (p_i-\avg{n_c}_i) p_i q_i C_2]
\end{align}
Where $C_2$ is
\begin{align}
    C_2&=\frac{\partial C_1}{\partial \beta^*}=\frac{\partial}{\partial \beta^*} \frac{\sum_i p_i q_i \epsilon_i}{\sum_i p_i q_i}\nonumber\\
    &=\frac{\frac{\partial }{\partial \beta^*}(\sum_i \epsilon_i q_i p_i) \sum_i p_i q_i - \sum_i p_i q_i \epsilon_i \frac{\partial}{\partial \beta^*}(\sum_i p_i q_i)}{(\sum_i p_i q_i)^2}\nonumber\\
    &=\frac{[\sum_i \epsilon_i(-\epsilon_i p_i q_i)q_i + \sum_i \epsilon_i p_i (p_i q_i \epsilon_i)]\sum_k p_k q_k-\sum_k \epsilon_k p_k q_k[\sum_i \epsilon_i p_i q_i p_i+\sum_i q_i(-\epsilon_i p_i q_i)]}{(\sum_i p_i q_i)^2}\nonumber\\
    &=\frac{[-\epsilon_i^2p_i q_i^2+ \sum_i \epsilon_i^2 p_i^2 q_i]\sum_k p_k q_k-\sum_k \epsilon_k p_k q_k[\sum_i \epsilon_i p_i^2 q_i-\sum_i \epsilon_i p_i q_i^2]}{(\sum_i p_i q_i)^2}\nonumber\\
    &=\frac{\sum_i(\epsilon_i^2 p_i q_i(p_i-q_i))}{\sum_i p_i q_i}-\frac{\sum_k \epsilon_k p_k q_k}{(\sum_i p_i q_i)^2}(\sum_i \epsilon_i p_i q_i (p_i-q_i))\nonumber\\
    &=\frac{\sum_i(\epsilon_i^2 p_i q_i(p_i-q_i))}{\sum_i p_i q_i}-\frac{C_1(\sum_i \epsilon_i p_i q_i (p_i-q_i))}{\sum_i p_i q_i}\nonumber\\
    &=\frac{\sum_i \epsilon_i p_i q_i(p_i - q_i)(\epsilon_i-C_1)}{\sum_i p_i q_i}
\end{align}
For the other term, it can observed that the only difference between the $\beta^*$ derivative and $y$ is a factor of $-\epsilon_i$. Using this knowledge, the other term can be easily written as,
\begin{align}
    \frac{\partial}{\partial y}\qty{\frac{d\chi^2}{d\beta^*}}&=2 
    \sum_i (p_i-\avg{n_c}_i)\Biggr[(\epsilon_i-C_1)p_iq_i[p_i-q_i-p_iq_i]+p_iq_iC_3\Biggr]
\end{align}
where $C_3$ is (using the same factor of $-\epsilon_i$),
\begin{gather}
    C_3=\frac{\partial C_1}{\partial y}=\frac{\partial}{\partial 
    y}\qty{\frac{\sum_i \epsilon_i q_i p_i}{\sum_i p_i q_i}}=\frac{\sum_i p_i q_i(q_i-p_i)(\epsilon_i-C_1)}{\sum_i p_i q_i}
\end{gather}
With this, the full second derivative is
\begin{align}
&\frac{d}{d\beta^*} \qty{\frac{d \chi^2}{d\beta^*}} = \frac{\partial }{\partial \beta^*} \eval{\qty{\frac{d \chi^2}{d\beta^*}}}_y+ \frac{\partial }{\partial y} \eval{\qty{\frac{d \chi^2}{d\beta^*}}}_{\beta^*} C_1\nonumber\\
&\quad=2\sum_i\qty{(\epsilon_i-C_1)^2[(p_i-\avg{n_c}_i)p_iq_i(q_i-p_i)+p_i^2q_i^2]+(p_i-\avg{n_c}_i)p_iq_i(C_1C_3+C_2)}
\end{align}